\section{Discussion}
After analysing the dataset, the group is a bit concerned if the dataset is big enough. With only around 500 records and 11 attributes, the dataset is a bit small. The boxplot shows that there is a lot of outliers in the dataset and no concrete data class where to be found in the dataset, which made the pc-analyse very hard. It means we should exclude the outliers after we move on the next step on our dataset.Also,some of the attributes are kind of skewed, we used some transform to pre-process them.Among all the attributes, FFMC, ISI and temp are considering normal distributed, which can be esaily to analyze in the further work. After calculated the C\^{O}R value, we found out that the correltation between attributes are not that abvious. Only two coulpe of attributes are kind of correltated.It is also the aim of our machine learning task, to find out more correlations during our study.  

Before  visualizate the dataset, we made assumption that area and tempreture were highly correltive with the forest fire.Though the tendency isn't that obivous, they will still be the attributes we are more interested in the further study, since our orignal aim of learning this dataset is help to prevent forest fire.Besides, the previously work on the dataset give us an idea to combine different attribute as new element to analyze. 