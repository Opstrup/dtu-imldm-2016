\section{Description of the data set}

\subsection*{Problem of interest}
The data is about forest fires. The goal of the data is to try to predict forest fries in an attempt to prevent casualties and property damages.

\subsection*{Place of data}
The data is obtained from this link: http://archive.ics.uci.edu/ml/datasets/Forest+Fires

\subsection*{Previously work on the data set}
P. Cortez and A. Morais. A Data Mining Approach to Predict Forest Fires using Meteorological Data.
In Proceedings of the 13th EPIA 2007 - Portuguese Conference on Artificial Intelligence, 
December, 2007. (http://www.dsi.uminho.pt/~pcortez/fires.pdf) \newline

In the above reference, the output "area" was first transformed with a ln(x+1) function.
Then, several Data Mining methods were applied. After fitting the models, the outputs were
post-processed with the inverse of the ln(x+1) transform. Four different input setups were
used. The experiments were conducted using a 10-fold (cross-validation) x 30 runs. Two
regression metrics were measured: MAD and RMSE. A Gaussian support vector machine (SVM) fed
with only 4 direct weather conditions (temp, RH, wind and rain) obtained the best MAD value:
12.71 +- 0.01 (mean and confidence interval within 95\% using a t-student distribution). The
best RMSE was attained by the naive mean predictor. An analysis to the regression error curve
(REC) shows that the SVM model predicts more examples within a lower admitted error. In effect,
the SVM model predicts better small fires, which are the majority.

\subsection*{Primary machine learning modeling aim}
Detection and test of outlier methods and try different regression methods and look at the correlation between the temperature, wind, rain and the burn area. 
